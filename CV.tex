publipu%%%%%%%%%%%%%%%%%%%%%%%%%%%%%%%%%%%%%%%%%
% Wilson Resume/CV
% XeLaTeX Template
% Version 1.0 (22/1/2015)
%
% This template has been downloaded from:
% http://www.LaTeXTemplates.com
%
% Original author:
% Howard Wilson (https://github.com/watsonbox/cv_template_2004) with
% extensive modifications by Vel (vel@latextemplates.com)
%
% License:
% CC BY-NC-SA 3.0 (http://creativecommons.org/licenses/by-nc-sa/3.0/)
%
%%%%%%%%%%%%%%%%%%%%%%%%%%%%%%%%%%%%%%%%%

%----------------------------------------------------------------------------------------
%	PACKAGES AND OTHER DOCUMENT CONFIGURATIONS
%----------------------------------------------------------------------------------------

\documentclass[10pt]{article} % Default font size

\input{structure.tex} % Include the file specifying document layout

%----------------------------------------------------------------------------------------

\begin{document}

%----------------------------------------------------------------------------------------
%	NAME AND CONTACT INFORMATION
%----------------------------------------------------------------------------------------

\title{Léopold Crestel -- Ph.D student at IRCAM} % Print the main header

%------------------------------------------------

\parbox{0.5\textwidth}{ % First block
\begin{tabbing} % Enables tabbing
\hspace{3cm} \= \hspace{4cm} \= \kill % Spacing within the block
{\bf Address} \> 6 Rue du Saint-Gothard,\\ % Address line 1
\> Paris, 75014\\ % Address line 2
{\bf Date of Birth} \> 21$^{st}$ June 1991\\ % Date of birth 
{\bf Nationality} \> French% Nationality
\end{tabbing}}
\hfill % Horizontal space between the two blocks
\parbox{0.5\textwidth}{ % Second block
\begin{tabbing} % Enables tabbing
\hspace{3cm} \= \hspace{4cm} \= \kill % Spacing within the block
%{\bf Home Phone} \> +0 (000) 111 1111 \\ % Home phone
{\bf Mobile Phone} \> +033 7 78 67 17 45\\ % Mobile phone
{\bf Email} \> \href{mailto:leopold.crestel@ircam.fr}{leopold.crestel@ircam.fr} \\ % Email address
\end{tabbing}}

%----------------------------------------------------------------------------------------
%	PERSONAL PROFILE
%----------------------------------------------------------------------------------------

\section{Research interest}
Machine learning, neural networks, sequential models, multi-modal models, automatic musical orchestration.
%----------------------------------------------------------------------------------------
%	EDUCATION SECTION
%----------------------------------------------------------------------------------------

\section{Education}
\tabbedblock{
\bf{2014-2015} \> Master ATIAM - \href{http://www.atiam.ircam.fr/}{IRCAM, UPMC and Télécom Paristech} - Paris \\[5pt]
\>  Master in Acoustic, Signal processing and Computer sciences applied to Music\\
}

%------------------------------------------------

\tabbedblock{
\bf{2011-2015} \> Engineering degree - Télécom Paristech, Paris\\[5pt]
\> Major in audio signal processing\\
\> Minor in Statistics and Probability\\
\> Cumulative GPA - 3.80/4.0\\
}

\tabbedblock{
\bf{2009-2011} \> Preparatory classes - Lycée Faidherbe, Lille\\[5pt]
\> MPSI/MP*\\
}


%----------------------------------------------------------------------------------------
%	EMPLOYMENT HISTORY SECTION
%----------------------------------------------------------------------------------------

\section{Professional experience}

\job
{Oct. 2015 -}{Sept. 2018}
{IRCAM, 1 Place Igor Stravinsky, Paris 75014, France}
{}
{Ph.D student}
{Thesis subject : Deep symbolic learning of multiple temporal granularities for musical orchestration. The goal of the PhD project is to provide an approach that could help in translating the intent of a composer in the process of orchestration. Hence, the main idea is to first learn the inherent structures that co-exist between different musical elements (relationships inside the symbolic knowledge of musical scores, between different signals but also between the signal and the score). Then, based on the learned connexionnist architectures of representation, the system could propose some re-orchestration and original improvisations. Modelling orchestral time series is the massive challenge at the hearth of this project. It raises several important issues for the deep learning field : how to model multi-modal data, highly sparse and with multiple temporal granularities.}


\job
{Feb. 2015 -}{July 2015}
{IRCAM, 1 Place Igor Stravinsky, Paris 75014, France}
{}
{Reasearch Intern}
{Deep symbolic learning for musical orchestration analysis and generation
\begin{itemize-noindent}
\item{Development of a music generation model based on a conditional RBM. Evaluation in a quantitative framework based on a predictive task}
\item{Extension of the model to an automatic orchestration system}
\item{Definition of a quantitative evaluation framework for the orchestration based on a predictive task}
\item{Realisation with the help of team-mates of a real-time generative orchestration system}
\end{itemize-noindent}
}

\job
{Feb. 2014 -}{Aug. 2014}
{Arkamys, 31 rue Pouchet, Paris 75017, France }
{}
{Innovation Intern}
{
Active reduction of the engine noise in the cockpit of a car 
\begin{itemize-noindent}
\item{Recommendation for the architecture of the final system}
\item{Development of an active noise reduction prototype with 2 loudspeakers, a microphone and a Digital Signal Processing board}
\end{itemize-noindent}
}


\job
{Sept. 2013 -}{Feb. 2014}
{Universitat Pompeu Fabra - Music Technology Group, Roc Boronat 138, Barcelona 08018, Spain}
{}
{Research intern}
{Work on \textit{Kaleivoicecope}, a voice transformation module based on Wide-Band Harmonic Sinusoidal Modelling
\begin{itemize-noindent}
\item{Realization of a perceptive test to determine the most relevant transformation in order to perfomr gender transformation}
\end{itemize-noindent}
}

\section{Publication and award}
\simple{\begin{itemize-noindent}
\item{Doctor Research Scholarship, EDITE, 2015}
\item{M thesis  + Article ??}
\end{itemize-noindent}
}


%----------------------------------------------------------------------------------------
%	IT/COMPUTING SKILLS SECTION
%----------------------------------------------------------------------------------------

\section{Programming}
\simple{\begin{itemize-noindent}
\item{Python/Theano, LUA/Torch, Matlab}
\item{C, C++, Java}
\item{HTML, CSS, PHP, JS}
\end{itemize-noindent}
}

\section{Complementary formation}
\tabbedblock{
\bf{2013-2016} \> Certificate  of music - Conservatoire du $7^{e}$ arrondissement, Paris\\[5pt]
\> Major in piano interpretation\\
\> Minor in piano jazz and electroacoustic composition
}
\tabbedblock{
\bf{1998-2009} \> End-of-study diploma in Piano - Conservatoire National de Région, Douai \\[5pt]
\>  Classical music\\
}

%----------------------------------------------------------------------------------------
%	INTERESTS SECTION
%----------------------------------------------------------------------------------------

\end{document}